\clearpage
\section{Bài 1}
\subsection{Vấn đề}
Viết một chương trình C++ để quản lý thông tin nhân viên trong một công ty. Sử dụng
tính kế thừa để tạo lớp cơ sở NhanVien và hai lớp con Quản lý và Kỹ sư. Mỗi loại nhân
viên có các thuộc tính như sau:
\begin{itemize}
    \item Nhân viên: Mã số nhân viên, Tên, Lương cơ bản.
    \item Quản lý: Thừa kế từ Nhân viên với thêm thuộc tính là Tỷ lệ thưởng. Phương thức TienThuong () tính toán tiền thưởng dựa trên tỷ lệ thưởng và lương cơ bản.
    \item Kỹ sư: Thừa kế từ Nhân viên với thêm thuộc tính là Số giờ làm thêm. Phương
thức TienThuong () tính toán tiền thưởng dựa trên số giờ làm thêm (mỗi giờ làm
thêm được trả 100.000).
\end{itemize}
\textbf{Lưu ý: Đảm bảo rằng chương trình có khả năng nhập và hiển thị thông tin đúng cho cả quản lý và kỹ sư.}
\newline 
\textbf{Gợi ý:} 
\begin{itemize}
    \item Sử dụng các lớp và tính kế thừa để cấu trúc dữ liệu nhân viên.
    \item Sử dụng phương thức TienThuong() để tính toán tiền thưởng.
    \item Sử dụng phương thức Xuat() để hiển thị thông tin.
\end{itemize}
\subsection{Class Diagram}
\includegraphics[width = 14cm]{graphics/cd1.png}
\subsection{Code}
\item {\textbf{Lớp Nhân Viên: }}
\begin{minted}[frame=lines, linenos, breaklines]{c++}
class NhanVien{
protected:
    string MaSo, Ten;
    double Luong;
public:
    NhanVien() {} 
    NhanVien(string MaSo, string Ten, double Luong) : MaSo(MaSo), Ten(Ten), Luong(Luong) {}
    virtual ~NhanVien() {}
    virtual double TienThuong() const = 0;
    virtual void Xuat() const {
        cout << "Nhan Vien: " << Ten << "\n";
        cout << "MaSo: " << MaSo << "\n";
        cout << "Luong co ban: " << fixed << setprecision(2) << Luong << "\n";
    }
};
\end{minted}

\item{\textbf{Lớp Quản Lý: }}
\begin{minted}[frame=lines, lineos, breaklines]{c++}
    class QuanLy : public NhanVien{
private: 
    double TyLeThuong;
public:
    QuanLy() {}
    QuanLy(string MaSo, string Ten, double Luong, double TyLeThuong) : NhanVien(MaSo, Ten, Luong), TyLeThuong(TyLeThuong) {}
    double TienThuong() const override {
        return TyLeThuong * Luong;
    }
    void Xuat() const override{ 
        cout << "Đay la thong tin cua Quan Ly" << "\n";
        NhanVien::Xuat();
        //cout << "Ty le thuong: " << TyLeThuong << "\n";
        cout << "Tien thuong: " << TienThuong() << "\n";
        cout << "---------------------------" << "\n";
    }
};
\end{minted}
\item{\textbf{Lớp Kỹ sư: }}
\begin{minted}[frame=lines, linenos, breaklines]{c++}
class KySu : public NhanVien{
private:
    int SoGioLamThem;
public:
    KySu() {}
    KySu(string MaSo, string Ten, double Luong, int SoGioLamThem) : NhanVien(MaSo, Ten, Luong), SoGioLamThem(SoGioLamThem) {}
    double TienThuong() const override {
        return (double) 100000.0 * SoGioLamThem;
    }
    void Xuat() const override{
        cout << "Đay la thong tin cua Ky Su" << "\n";
        NhanVien::Xuat();
        //cout << "So gio lam them: " << SoGioLamThem << "\n";
        cout << "Tien Thuong: " << TienThuong() << "\n";
        cout << "---------------------------" << "\n";
    }
};
\end{minted}
\item{\textbf{Hướng giải quyết:}}
\begin{itemize}
    \item Lớp Quản lý và lớp Kỹ Sư kế thừa lớp Nhân Viên
    \item Sử dụng phương thức TienThuong() để tính toán tiền thưởng.
    \item Sử dụng phương thức Xuat() để hiển thị thông tin.
    \item Sử dụng cơ chế đa hình (virtual) để gọi phương thức phù hợp.
\end{itemize}
\item{\textbf{Gọi hàm trong main:}}
\begin{minted}[frame=lines, linenos, breaklines]{c++} 
#include "quanly.h"
#include "kysu.h"

string ms, ten;
double luong, tlt;
int sogio;
QuanLy NhapQuanLy() {
    cout << "Nhap thong tin quan ly: " << "\n";
    cout << "MaSo: "; 
    cin >> ms;
    cout << "Ten: ";
    cin >> ten;
    cout << "Luong co ban: ";
    cin >> luong;
    cout << "Ty Le Thuong: ";
    cin >> tlt;
    QuanLy a(ms, ten, luong, tlt);
    return a;
}

KySu NhapKySu() {
    cout << "Nhap thong tin ky su: " << "\n";
    cout << "MaSo: "; 
    cin >> ms;
    cout << "Ten: ";
    cin >> ten;
    cout << "Luong co ban: ";
    cin >> luong;
    cout << "So gio lam them: ";
    cin >> sogio;
    KySu a(ms, ten, luong, sogio);
    return a;
}
signed main() {

    QuanLy a = NhapQuanLy();
    KySu b = NhapKySu();
    cout << "---------------------------" << "\n";
    a.Xuat();
    b.Xuat();    
    return 0;

}
}
\end{minted}
\subsection{Kiểm thử các test case}
\includegraphics{graphics/test1.png}
