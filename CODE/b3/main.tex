\clearpage
\section{Bài 3}
\subsection{Vấn đề}
Một nông trại chăn nuôi có 3 loại gia súc: bò, cừu, và dê. Mỗi loại gia súc
đều có thể sinh con, cho sữa và phát ra tiếng kêu riêng của chúng. Khi đói, các gia súc
sẽ phát ra tiếng kêu để đòi ăn. Sau một thời gian chăn nuôi, người chủ nông trại muốn
thống kê xem trong nông trại có bao nhiêu gia súc ở mỗi loại, tổng số sữa mà tất cả các
gia súc của ông đã cho. \newline 
Áp dụng kế thừa, xây dựng chương trình cho phép người chủ nông trại nhập vào số
lượng gia súc ban đầu ở mỗi loại.
\begin{itemize}
    \item a) Một hôm người chủ nông trại đi vắng, tất cả gia súc trong nông trại đều đói. Hãy
cho biết những tiếng kêu nghe được trong nông trại.
    \item b) Chương trình sẽ đưa ra thống kê các thông tin người chủ mong muốn (nêu trên)
sau một lứa sinh và một lượt cho sữa của tất cả gia súc. Biết rằng:
    \begin{itemize}
        \item Tất cả gia súc ở mỗi loại đều sinh con.
        \item Số lượng sinh của mỗi gia súc là ngẫu nhiên.
        \item Tất cả gia súc mỗi loại đều cho sữa.
        \item Số lượng sinh của mỗi gia súc cho sữa là ngẫu nhiên nhưng trong giới hạn sau:
        \begin{itemize}
            \item Bò: 0 – 20 lít
            \item Cừu: 0 – 5 lít
            \item Dê: 0 – 10 lít
        \end{itemize}
    \end{itemize}
\end{itemize}

\subsection{Class Diagram}
\includegraphics[width = 16cm]{graphics/cd3.png}
\subsection{Code}
\item {\textbf{Lớp Gia Súc: }}
\begin{minted}[frame=lines, linenos, breaklines]{c++}
#ifndef GIASUC_H
#define GIASUC_H

#include <bits/stdc++.h> 
using namespace std;

class GiaSuc{
protected:
    int SoLuong;
public: 
    GiaSuc(int SoLuong) : SoLuong(SoLuong) {}
    virtual ~GiaSuc() {} 
    virtual string Sound() const = 0;
    virtual int SoLuongSinh () const = 0;
    virtual int SoLuongSua() const = 0;

    int getSoLuong() const {
        return SoLuong;
    }

    void tangSoLuong(int them) {
        SoLuong += them;
    }
};
#endif
\end{minted}

\item {\textbf{Lớp Bò: }}
\begin{minted}[frame=lines, linenos, breaklines]{c++}
#ifndef BO_H
#define BO_H

#include "giasuc.h"

class Bo : public GiaSuc{

public: 
    Bo(int SoLuong) : GiaSuc(SoLuong) {}
    string Sound() const override {
        return "Boooooooooo";
    }

    int SoLuongSinh() const override {
        return rand() % 3 + 1;
    }
    int SoLuongSua() const override {
        return rand() % 21;
    }
};
#endif
\end{minted}

\item {\textbf{Lớp Cừu: }}
\begin{minted}[frame=lines, linenos, breaklines]{c++}
#ifndef CUU_H
#define CUU_H

#include "giasuc.h"

class Cuu : public GiaSuc{

public: 
    Cuu(int SoLuong) : GiaSuc(SoLuong) {}
    string Sound() const override {
        return "Cuuuuuuuuuuuuu";
    }
    int SoLuongSinh() const override {
        return rand() % 3 + 1;
    }
    int SoLuongSua() const override {
        return rand() % 6;
    }
};
#endif
\end{minted}

\item {\textbf{Lớp Dê: }}
\begin{minted}[frame=lines, linenos, breaklines]{c++}
#ifndef DE_H
#define DE_H

#include "giasuc.h"

class De : public GiaSuc{

public: 
    De(int SoLuong) : GiaSuc(SoLuong) {};
    string Sound() const override {
        return "Deeeeeeeeee";
    }
    int SoLuongSinh() const override {
        return rand() % 3 + 1;
    }
    int SoLuongSua() const override {
        return rand() % 11;
    }
};
#endif
\end{minted}
\item {\textbf{Lớp Nông Trại: }}
\begin{minted}[frame=lines, linenos, breaklines]{c++}
#ifndef NONGTRAI_H
#define NONGTRAI_H

#include "giasuc.h"

class NongTrai{
private: 
    vector <GiaSuc*> Cattle;
public: 
    ~NongTrai() {
        for (GiaSuc* gs : Cattle) {
            delete gs;
        }
    }
    void themGiaSuc(GiaSuc* gs) {
        Cattle.push_back(gs);
    }

    void tiengKeu() const {
        for (const GiaSuc* gs : Cattle) {
            cout << gs->Sound() << endl;
        }
    }

    void thongKe() {
        int tongSoLuong = 0;
        int tongSua = 0;

        for (GiaSuc* gs : Cattle) {
            int soLuongMoi = 0;
            int sua = 0;

            // Thống kê từng loại gia súc
            for (int i = 0; i < gs->getSoLuong(); ++i) {
                soLuongMoi += gs->SoLuongSinh();
                sua += gs->SoLuongSua();
            }

            gs->tangSoLuong(soLuongMoi);
            tongSoLuong += gs->getSoLuong();
            tongSua += sua;

            cout << "Loại gia súc: " << gs->Sound() << endl;
            cout << "Số lượng mới sinh: " << soLuongMoi << endl;
            cout << "Tổng sữa: " << sua << " lít" << endl;
        }

        cout << "\nTổng số lượng gia súc: " << tongSoLuong << endl;
        cout << "Tổng lượng sữa thu được: " << tongSua << " lít" << endl;
    }
};
#endif
\end{minted}
\item{\textbf{Hướng giải quyết:}}
\begin{itemize}
    \item Lớp Bò, Cừu, Dê kế thừa lớp Gia Súc
    \item Lớp Nông Trại tạo ra để quản lý tất cả số liệu, thống kê và tiếng kêu của tất cả các gia súc trong nông trại
\end{itemize}
\item{\textbf{Gọi hàm trong main:}}
\begin{minted}[frame=lines, linenos, breaklines]{c++} 
#include "bo.h"
#include "cuu.h" 
#include "de.h"
#include "nongtrai.h"

signed main()
{
    srand(time(NULL));
    int soBo, soDe, soCuu;
    cout << "Nhap so luong bo: "; cin >> soBo;
    cout << "Nhap so luong cuu: "; cin >> soCuu;
    cout << "Nhap so luong de: "; cin >> soDe;
    NongTrai nt;
    nt.themGiaSuc(new Bo(soBo));
    nt.themGiaSuc(new Cuu(soCuu));
    nt.themGiaSuc(new De(soDe));

    cout << "Tieng keu cua gia suc khi doi: " << "\n";
    nt.tiengKeu();

    cout << "Thong ke sau mot lua sinh va cho sua: " << "\n";
    nt.thongKe();
    return 0;
}
\end{minted}
\subsection{Kiểm thử các test case}
\includegraphics{graphics/test3.png}
